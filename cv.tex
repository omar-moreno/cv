%
% @file cv.tex
% @author Omar Moreno
%

\documentclass[11pt]{article}

%%%%%%%%%%%%%%%%
%   Packages   %
%%%%%%%%%%%%%%%%

% Enable extra colors
\usepackage[usenames,dvipsnames]{xcolor}

% Package that allows setting of page geometry e.g. margins
\usepackage[top=0.5in, bottom=0.5in, left=0.5in, right=0.5in]{geometry}

%
\usepackage{enumitem}

%
\usepackage[none]{hyphenat}

%
\usepackage[hidelinks]{hyperref}

%
\usepackage{fontspec}

%
\defaultfontfeatures{Extension = .otf}

%
\usepackage{fontawesome}

%
\usepackage{changepage}

% Remove page numbers
\pagenumbering{gobble}

% Use the font Computer Modern Sans Serif
\renewcommand*{\familydefault}{\sfdefault}

% Define Dim Gray color 
\definecolor{dimgray}{HTML}{696969}
\definecolor{indigodye}{HTML}{0D4F8B}

%
% resumesection : Command used to denote a new section of the CV
% 
% #1 Name of the section
%
\newcommand{\cvsection}[1] {
    \noindent
    \textcolor{indigodye}{\rule{.15\textwidth}{.1in} \hspace{0.01 \textwidth} \textbf{\Large{#1}}} \newline 
}

\newcommand{\educationentry}[5] { 
    \noindent
    \begin{minipage}[t]{0.15\textwidth} \begin{flushright} #1 \end{flushright} \end{minipage} \hspace{0.01\textwidth}
    \begin{minipage}[t]{0.84\textwidth} 
        \textbf{#2}, \emph{#3}, #4 \newline 
        #5 \newline
%        Advisor: #6 \newline 
    \end{minipage}
}

\newcommand{\experienceentrycollab}[6] { 
    \noindent
    \begin{minipage}[t]{0.15\textwidth} \begin{flushright} #1 \end{flushright} \end{minipage} \hspace{0.01\textwidth}
    \begin{minipage}[t]{0.84\textwidth} 
        \textbf{#2} \newline
        \emph{#3}, #4 \newline
        Collaborations: #5
    \end{minipage} \\[0.01pt]
    #6 \vspace{11pt}
}

\newcommand{\experienceentry}[5] { 
    \noindent
    \begin{minipage}[t]{0.15\textwidth} \begin{flushright} #1 \end{flushright} \end{minipage} \hspace{0.01\textwidth}
    \begin{minipage}[t]{0.84\textwidth} 
        \textbf{#2} \newline
        \emph{#3}, #4
    \end{minipage} \\[0.01pt]
    #5 \vspace{11pt}
}

\newcommand{\skillsentry}[2] { 
    \noindent
    \begin{minipage}[t]{0.15\textwidth} \begin{flushright} #1 \end{flushright} \end{minipage} \hspace{0.01\textwidth}
    \begin{minipage}[t]{0.84\textwidth} #2 \end{minipage}
}

\newcommand{\awardentry}[3] { 
    \noindent
    \begin{minipage}[t]{0.15\textwidth} \begin{flushright} #1 \end{flushright} \end{minipage} \hspace{0.01\textwidth}
    \begin{minipage}[t]{0.84\textwidth} #2,  \emph{#3} 
    \end{minipage}
}

\newcommand{\leadentry}[2] { 
    \noindent
    \begin{minipage}[t]{0.15\textwidth} \begin{flushright} #1 \end{flushright} \end{minipage} \hspace{0.01\textwidth}
    \begin{minipage}[t]{0.84\textwidth} #2
    \end{minipage}
}

\newcommand{\teachingentry}[4] { 
    \noindent
    \begin{minipage}[t]{0.15\textwidth} \begin{flushright} #1 \end{flushright} \end{minipage} \hspace{0.01\textwidth}
    \begin{minipage}[t]{0.84\textwidth}
        \textbf{#2} \newline
        #3  
    \end{minipage} \\[0.01pt]
    #4 \vspace{11pt}
}


\renewcommand{\section}[2]{}

\begin{document}
    
    %%%%%%%%%%%%%%
    %   Header   %
    %%%%%%%%%%%%%%
    \noindent
    \begin{minipage}[c]{0.5\textwidth}
        \begin{flushleft}
            \Huge{Omar Moreno} \newline
            \Large{\textcolor{dimgray}{Curriculum Vitae}}
        \end{flushleft}
    \end{minipage}
    \begin{minipage}[c]{0.50\textwidth}
        \begin{flushright}
            \color{dimgray} \em
            Los Altos, CA           \\
            \faMobilePhone \hspace{1pt} +1 (562) 396-1622       \\
            \faEnvelope \hspace{1pt} \href{mailto:omoreno@slac.stanford.edu}{omoreno@slac.stanford.edu}          \\
            \faLinkedin \hspace{1pt} \href{https://www.linkedin.com/in/omarmoreno2}{omarmoreno2} \\
            \faGithub \hspace{1pt} \href{https://github.com/omar-moreno}{omar-moreno}            \\
        \end{flushright}
    \end{minipage}

    \cvsection{Education}
        \educationentry{2016}
                       {Ph.D. in Physics}
                       {University of California at Santa Cruz}
                       {Santa Cruz, CA}
                       {Dissertation: Search for a Heavy Photon in the 2015 Engineering Run Data of the Heavy Photon Search Experiment}
%                       {John Jaros}
        \educationentry{2009}
                       {M.Sc. in Physics}
                       {California State University, Los Angeles}
                       {Los Angeles, CA}
                       {Thesis: Measurement of the Analyzing Power for the Reactions $p + CH_{2}
                       \rightarrow X$ at a \\ Proton Momentum of 2.2032 GeV/c}
%                       {Konrad Aniol} 
        \educationentry{2006}
                       {B.Sc. in Applied Physics}
                       {University of California at Irvine}
                       {Irvine, CA}
                       {Thesis: Search for the $B\rightarrow e^+e^-$ as a Hint to Supersymmetry}
%                       {David Kirkby}

    \cvsection{Research Experience}
        \experienceentrycollab{2016-present}
                        {Research Associate}
                        {SLAC National Accelerator Laboratory}
                        {Menlo Park, CA}
                        {Light Dark Matter eXperiment, Heavy Photon Search}
                        {   
                            \begin{itemize}[label=\textcolor{indigodye}{$\circ$}, noitemsep, nolistsep, leftmargin=0.19\textwidth]
                                \item Leading the development of a Geant4, C++ based simulation and reconstruction 
                                      framework for the Light Dark Matter eXperiment (LDMX).
                                \item Used machine learning techniques to identify and veto rare backgrounds (e.g.
                                      photo/electro-nuclear) expected by LDMX.
                                \item Managed the production and reconstruction of large ($\sim$10 billion) signal and
                                      background samples that were used to study the sensitivity of LDMX to several 
                                      physics scenarios.
                                \item Played a key role in both the comissioning of the Heavy Photon Search 
                                      (HPS) silicon vertex tracker (SVT) data acquisition system and its integration
                                      with JLab's Hall B central data acquisition system for the 2019 run.
                                \item Conducted a resonance search for a new gauge boson (dark photon) in the mass range 
                                      19 MeV/c$^2$ to 81 MeV/c$^2$ leading to the first physics publication by the
                                      Heavy Photon Search experiment.
                                      %utilizing frequentist statistical analysis and 
                                      %maximum likelihood techniques.
                                %\item Lead developer of the Heavy Photon Search Toolkit for Reconstruction framework
                                %      to be used for analysis of data collected in future runs.
                                \item Mentored HPS graduate students and post-docs on machine learning, 
                                      analysis and data acquisition projects.
                                \item Mentored LDMX undergraduates, graduate students and post-docs on 
                                      machine learning, simulation and reconstruction projects.
                            \end{itemize} 
                        } 
        \experienceentrycollab{2011-2016}
                        {Graduate Student Researcher}
                        {Santa Cruz Institute for Particle Physics}
                        {Santa Cruz, CA}
                        {Heavy Photon Search}
                        {   
                            \begin{itemize}[label=\textcolor{indigodye}{$\circ$}, noitemsep, nolistsep, leftmargin=0.19\textwidth]
                                \item Lead developer of both the analysis pipeline and statistical package
                                      used to conduct a resonance search for a new fundamental particle.
                                      %, 
                                      %dark photon, thought to mediate dark matter interactions.
                                    %$\item Utilized frequentist statistical analysis and maximum likelihood estimation to conduct a 
                              %      resonance search for a new fundamental particle, heavy photon, thought to mediate dark matter 
                              %      interactions.
                              %\item Applied machine learning techniques  
                              %      to reject trident and wide angle brem backgrounds.
                                    %using a Random Forest Algorithm in Scikit-Learn, boosting the 
                                    %signal/background fraction by 40\%.
                              \item Co-creator of both a C++ and Java analysis pipeline used to clean up and process
                                    data into physics objects.
                              %      and visualize over 10 TB of noisy data (detector output).
                              %\item Developed Java front end used to load and retrieve greater than 100,000 
                              %      calibration constants from a MySQL database.
                              \item Tested and commissioned the HPS SVT data acquisition system used for the 2015 and
                                    2016 engineering runs. 
                                    %used to continuously read out 23,004 channels at a rate of up to
                                    %50 kHz.
                              \item Characterized the performance of several 
                                    components of the HPS SVT including the front end readout boards and the silicon 
                                    microstrip modules at various stages of production.
                              \item Key member of team that assembled, installed and commissioned the HPS SVT.
                          \end{itemize} 
                      } 
        \experienceentrycollab{2009-2011}
                        {Graduate Student Researcher}
                        {Santa Cruz Institute for Particle Physics}
                        {Santa Cruz, CA}
                        {International Linear Collider}
                        {
                            \begin{itemize}[label=\textcolor{indigodye}{$\circ$}, noitemsep, nolistsep, leftmargin=0.19\textwidth]
                                \item Characterized the performance of the Long Shaping Time Front End readout chip at
                                      various stages of development.
                                  %\item Mentored several undergraduate students on various projects.
                            \end{itemize}
                        } 
        \experienceentrycollab{2007-2009}
                        {Graduate Student Researcher}
                        {Department of Physics and Astronomy, California State University, Los Angeles}
                        {Los Angeles, CA}
                        {GEP-III}
                        {
                            \begin{itemize}[label=\textcolor{indigodye}{$\circ$}, noitemsep, nolistsep, leftmargin=0.19\textwidth]
                                %\item Developed C++ analysis used to optimize detector selection criteria using regression
                                %      techniques and frequentist inference resulting in an improved understanding of the
                                %      interaction of the proton in polyethylene.
                                \item Used likelihood techniques to measure the form factor ratio, $G_{E_p}/G_{M_p}$, of
                                      the proton.
                                %using blind analysis techniques.
                            \end{itemize}
                        }  
        \experienceentrycollab{2005-2006}
                        {Undergraduate Researcher}
                        {Department of Physics and Astronomy, University of California, Irvine}
                        {Irvine, CA}
                        {BaBar}
                        { 
                            \begin{itemize}[label=\textcolor{indigodye}{$\circ$}, noitemsep, nolistsep, leftmargin=0.19\textwidth]
                                \item Developed analysis to measure the branching fraction for the extremely rare decay 
                                      $B\rightarrow e^+e^-$.
                                %using blind analysis and regression techniques.
                                \item Used a neural network to boost the identification of the particle decay 
                                      $\Lambda \rightarrow p \pi^-$ by 10\%.
                            \end{itemize}
                       } 
        \experienceentry{2000-2001}
                        {Mechanical Engineering Apprentice}
                        {Nasa Dryden Flight Research Center}
                        {Edwards, CA}
                        {
                            \begin{itemize}[label=\textcolor{indigodye}{$\circ$}, noitemsep, nolistsep, leftmargin=0.19\textwidth]
                                \item Designed and constructed a device used to evaluate the skin-friction reduction 
                                      of several Micro -Blowing Technique skins at supersonic speeds. 
                            \end{itemize}
                        } 
    
    \cvsection{Skills}
        \skillsentry{Prog. Languages}{Java, C++, C, Python, MySQL, XML, bash. Familiar with with HTML5 and Fortran}
        \skillsentry{Tools}{Linux, ROOT, Geant4, NumPy, matplotlib, scikit-learn, scipy, git, SVN, CMake, \LaTeX, RooFit, Mathematica}
        \skillsentry{Languages}{Fluent in English and Spanish}

    \cvsection{Appointments, Fellowships and Honors}
        \awardentry{2018}{Visiting Professor}{Università degli Studi di Sassari, Sassari, Italy}
        \awardentry{2012}{Margaret Burbidge Award for Best Experimental Research}{American Physical Society} 
        \awardentry{2011}{Regent's Fellowship}{University of California, Santa Cruz}
        \awardentry{2010}{GAANN Fellowship}{University of California, Santa Cruz}
        \awardentry{2009}{Special Recognition in Graduate Studies}
                         {California State University, Los Angeles}
        \awardentry{2009}{Margaziotis Award for Best Experimental Research}
                         {California State University, Los Angeles}
        \awardentry{2007-2009}{LSAMP Bridge to the Doctorate Fellowship}
                              {National Science Foundation} 
        \awardentry{2006}{California Alliance for Minority Participation Mentor of the Year}
                         {University of California, Irvine}
        \awardentry{2006}{Special Merit in Research}
                         {University of California, Irvine}
        \awardentry{2001-2002}{Chancellor's Leadership Scholar}
                         {University of California, Irvine}

    \cvsection{Leadership}
        \leadentry{2016-present}{Coordinator of the LDMX Software and Computing Working Group}
        \leadentry{2016-2017}{Leader of the HPS Tracking Working Group}
        \leadentry{2015-2018}{Leader of the Resonance Search Working Group}

    \cvsection{Teaching Experience}
        \teachingentry{2013-2015}{GRE Physics Bootcamp Instructor}
                      {Department of Physics, University of California, Santa Cruz}
                      {
                        \begin{itemize}[label=\textcolor{indigodye}{$\circ$}, noitemsep, nolistsep, leftmargin=0.19\textwidth]
                            \item Taught undergraduate level quantum mechanics.                                 
                          \end{itemize}
                      }

        \teachingentry{2009-2011}{Graduate Teaching Assistant}
                      {Department of Physics, University of California, Santa Cruz}
                      {
                        \begin{itemize}[label=\textcolor{indigodye}{$\circ$}, noitemsep, nolistsep, leftmargin=0.19\textwidth]
                            \item Physics 6A - Mechanics                                 
                            \item Physics 6B - Waves and Thermodynamics                  
                            \item Physics 6C - Electricity and Magnetism                
                          \end{itemize}
                      }

        \teachingentry{2007}{Graduate Teaching Assistant}
                      {Department of Physics and Astronomy, California State University, Los Angeles}
                      {  
                        \begin{itemize}[label=\textcolor{indigodye}{$\circ$}, noitemsep, nolistsep, leftmargin=0.19\textwidth]
                            \item Physics 211 - Classical Mechanics                      
                            \item Physics 213 - Electricity and Magnetism 
                        \end{itemize}
                      }
    
    \cvsection{Invited Talks}

    \noindent
    [1] O. Moreno, Accelerating Dark Matter,
        \emph{Dipartimento di Fisica, Università di Roma Tor Vergata}, 2018 
    
    \noindent
    [2] O. Moreno, Accelerating Dark Matter,
        \emph{Dipartimento di Fisica, Università degli Studi di Torino}, 2018 
    
    \noindent
    [3] O. Moreno, Accelerating Dark Matter,
        \emph{Dipartimento di Fisica, Università degli Studi di Genova}, 2018 

    \noindent
    [4] O. Moreno, Shedding Light on Dark Matter, 
        \emph{Dipartamento di Chimica, Università degli Studi di Sassari}, 2018  

    \noindent
    [5] O. Moreno, The Heavy Photon Search Experiment, 
        \emph{LHC Physics Center Topic Of The Week}, 2018 

    \noindent
    [6] O. Moreno, First Results from the Heavy Photon Search, 
        \emph{JLab Physics Seminar}, 2017 
    
    \noindent
    [7] O. Moreno, The Heavy Photon Search Experiment, 
        \emph{U.S. Cosmic Visions: New Ideas in Dark Matter}, 2017 \\

    %\newpage
    \cvsection{Publications}
    \vspace{-22pt}

    \begin{thebibliography}{99}

        \bibitem{Adrian:2018scb} 
            P.~H.~Adrian {\it et al.} [HPS Collaboration],
            ``Search for a dark photon in electroproduced $e^{+}e^{-}$ pairs with the Heavy Photon Search experiment at JLab,''
            Phys.\ Rev.\ D {\bf 98}, no. 9, 091101 (2018)
            doi:10.1103/PhysRevD.98.091101
            [arXiv:1807.11530 [hep-ex]].
        
        \bibitem{Akesson:2018vlm} 
            T.~Åkesson {\it et al.} [LDMX Collaboration],
            ``Light Dark Matter eXperiment (LDMX),''
            arXiv:1808.05219 [hep-ex].

        \bibitem{Puckett:2017flj} 
            A.~J.~R.~Puckett {\it et al.},
            ``Polarization Transfer Observables in Elastic Electron Proton Scattering at $Q^2 = $2.5, 5.2, 6.8, and 8.5 GeV$^2$,''
            Phys.\ Rev.\ C {\bf 96}, no. 5, 055203 (2017)
            Erratum: [Phys.\ Rev.\ C {\bf 98}, no. 1, 019907 (2018)]
            doi:10.1103/PhysRevC.98.019907, 10.1103/PhysRevC.96.055203
            [arXiv:1707.08587 [nucl-ex]].

        \bibitem{Puckett:2017egz} 
            A.~J.~R.~Puckett {\it et al.},
            ``Technical Supplement to "Polarization Transfer Observables in Elastic Electron-Proton Scattering at Q$^2$ = 2.5, 5.2, 6.8, and 8.5 GeV$^2$",''
            Nucl.\ Instrum.\ Meth.\ A {\bf 910}, 54 (2018)
            doi:10.1016/j.nima.2018.09.022
            [arXiv:1707.07750 [nucl-ex]].

        \bibitem{Battaglieri:2014hga} 
            M.~Battaglieri {\it et al.},
            ``The Heavy Photon Search Test Detector,''
            Nucl.\ Instrum.\ Meth.\ A {\bf 777}, 91 (2015)
            doi:10.1016/j.nima.2014.12.017
            [arXiv:1406.6115 [physics.ins-det]].

        \bibitem{Luo:2011uy} 
            W.~Luo {\it et al.} [GEp-III and GEp2gamma Collaborations],
            ``Polarization components in $\pi^{0}$ photoproduction at photon energies up to 5.6 GeV,''
            Phys.\ Rev.\ Lett.\  {\bf 108}, 222004 (2012)
            doi:10.1103/PhysRevLett.108.222004
            [arXiv:1109.4650 [nucl-ex]].

        \bibitem{Meziane:2010xc} 
            M.~Meziane {\it et al.} [GEp2gamma Collaboration],
            ``Search for effects beyond the Born approximation in polarization transfer observables in $\vec{e}p$ elastic scattering,''
            Phys.\ Rev.\ Lett.\  {\bf 106}, 132501 (2011)
            doi:10.1103/PhysRevLett.106.132501
            [arXiv:1012.0339 [nucl-ex]].

        \bibitem{Puckett:2010ac} 
            A.~J.~R.~Puckett {\it et al.},
            ``Recoil Polarization Measurements of the Proton Electromagnetic Form Factor Ratio to Q$^2$ = 8.5 GeV$^2$,''
            Phys.\ Rev.\ Lett.\  {\bf 104}, 242301 (2010)
            doi:10.1103/PhysRevLett.104.242301
            [arXiv:1005.3419 [nucl-ex]].

    \end{thebibliography}
    \nocite{*}

\end{document}
