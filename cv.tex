%
% cv.tex
% Omar Moreno
%

\documentclass[11pt]{article}

%%%%%%%%%%%%%%%%
%   Packages   %
%%%%%%%%%%%%%%%%

% Enable extra colors
\usepackage[usenames,dvipsnames]{xcolor}

% Package that allows setting of page geometry e.g. margins
\usepackage[top=0.5in, bottom=0.5in, left=0.75in, right=0.75in]{geometry}

%
\usepackage{enumitem}

%
\usepackage[none]{hyphenat}

%
\usepackage[hidelinks]{hyperref}

%
\usepackage{fontspec}

%
\defaultfontfeatures{Extension = .otf}

%
\usepackage{fontawesome}

%
\usepackage{changepage}

% Remove page numbers
\pagenumbering{gobble}

% Use the font Computer Modern Sans Serif
\renewcommand*{\familydefault}{\sfdefault}

% Define some colors
\definecolor{dimgray}{HTML}{696969}
\definecolor{indigodye}{HTML}{0D4F8B}
\definecolor{lighterblue}{HTML}{1766A6}

%
% resumesection : A CV section
% 
% #1 Name of the section
%
\newcommand{\cvsection}[1] {
    \noindent
    \textcolor{indigodye}{\rule{.15\textwidth}{.1in} \hspace{0.01 \textwidth} \textbf{\Large{#1}}} \newline 
}

%
% educationentry : An entry in the "Education" section.
%
% #1 Year degree received
% #2 Degree type
% #3 Institution
% #4 City, State
% #5 Thesis, Disseration title
%
\newcommand{\educationentry}[5] { 
    \noindent
    \begin{minipage}[t]{0.15\textwidth} \begin{flushright} #1 \end{flushright} \end{minipage} \hspace{0.01\textwidth}
    \begin{minipage}[t]{0.84\textwidth} 
        \textbf{#2}, \emph{#3}, #4 
%        #5 \newline
%        Advisor: #6 \newline 
    \end{minipage}
}

%
% experienceentry : An entry in the "Research Experience" section.
%
%
\newcommand{\experienceentry}[5] { 
    \noindent
    %\begin{minipage}[t]{0.15\textwidth} ~ \end{minipage} \hspace{0.01\textwidth}
    \begin{minipage}[t]{0.84\textwidth} 
      \textbf{#1, #2}
    \end{minipage} \\[0.01pt]
    \begin{minipage}[t]{0.15\textwidth} \begin{flushright} #3 \end{flushright} \end{minipage} \hspace{0.01\textwidth}
    \begin{minipage}[t]{0.84\textwidth} 
      \textbf{\textcolor{indigodye}{#4}} \vspace{5pt} %\newline
    \end{minipage} 
    %\\[0.01pt]
    #5 \vspace{11pt}
}

\newcommand{\skillsentry}[2] { 
    \noindent
    \begin{minipage}[t]{0.15\textwidth} \begin{flushright} #1 \end{flushright} \end{minipage} \hspace{0.01\textwidth}
    \begin{minipage}[t]{0.84\textwidth} #2 \end{minipage}
}

\newcommand{\awardentry}[3] { 
    \noindent
    \begin{minipage}[t]{0.15\textwidth} \begin{flushright} #1 \end{flushright} \end{minipage} \hspace{0.01\textwidth}
    \begin{minipage}[t]{0.84\textwidth} #2,  \emph{#3} 
    \end{minipage}
}

\newcommand{\leadentry}[2] { 
    \noindent
    \begin{minipage}[t]{0.15\textwidth} \begin{flushright} #1 \end{flushright} \end{minipage} \hspace{0.01\textwidth}
    \begin{minipage}[t]{0.84\textwidth} #2
    \end{minipage}
}

\newcommand{\teachingentry}[4] { 
    \noindent
    \begin{minipage}[t]{0.15\textwidth} \begin{flushright} #1 \end{flushright} \end{minipage} \hspace{0.01\textwidth}
    \begin{minipage}[t]{0.84\textwidth}
        \textbf{#2} \newline
        #3  
    \end{minipage} \\[0.01pt]
    #4 \vspace{11pt}
}


\renewcommand{\section}[2]{}

\begin{document}
    
    %%%%%%%%%%%%%%
    %   Header   %
    %%%%%%%%%%%%%%
    \noindent
    \begin{minipage}[c]{0.5\textwidth}
        \begin{flushleft}
            \Huge{Omar \textcolor{indigodye}{\textbf{Moreno}}} \newline
            \Large{\textcolor{dimgray}{Curriculum Vitae}}
        \end{flushleft}
    \end{minipage}
    \begin{minipage}[c]{0.50\textwidth}
        \begin{flushright}
            \color{dimgray} \em
            Scotts Valley, CA           \\
            \faMobilePhone \hspace{1pt} +1 (562) 396-1622       \\
            \faEnvelope \hspace{1pt} \href{mailto:omoreno@slac.stanford.edu}{omoreno@slac.stanford.edu}          \\
            \faLinkedin \hspace{1pt} \href{https://www.linkedin.com/in/omarmoreno2}{omarmoreno2} \\
            \faGithub \hspace{1pt} \href{https://github.com/omar-moreno}{omar-moreno}            \\
        \end{flushright}
    \end{minipage}

    \cvsection{Education}
        \educationentry{2016}
                       {Ph.D. in Physics}
                       {University of California at Santa Cruz}
                       {Santa Cruz, CA}
                       {Dissertation: Search for a Heavy Photon in the 2015 Engineering Run Data of the Heavy Photon Search Experiment}
        \educationentry{2009}
                       {M.Sc. in Physics}
                       {California State University, Los Angeles}
                       {Los Angeles, CA}
                       {Thesis: Measurement of the Analyzing Power for the Reactions $p + CH_{2} \rightarrow X$ at a Proton Momentum of 2.2032 GeV/c}
        \educationentry{2006}
                       {B.Sc. in Applied Physics}
                       {University of California at Irvine}
                       {Irvine, CA}
                       {Thesis: Search for the $B\rightarrow e^+e^-$ as a Hint to Supersymmetry}

    \cvsection{Research Experience}
        \experienceentry{SLAC National Accelerator Laboratory}{Menlo Park, CA}
                        {2020-present \\ 2016-2020}
                        {Project Scientist \\ Research Associate}
                        {                   
            \begin{itemize}[label=\textcolor{indigodye}{$\circ$}, noitemsep, nolistsep, leftmargin=0.19\textwidth]
                \item Software and Computing coordinator for the Light Dark Matter eXperiment (LDMX).
                \item Lead the development of a C++/python based simulation and reconstruction 
                  pipeline used to generate large Monte Carlo data sets (~$\sim$10 billion events per set) crucial
                  to understanding performance of the LDMX detector concept.
                \item Spearheaded the development of a data aquisition framework using C++ and used for 
                      configuring, operating and collecting data from the LDMX test beam detector.
                %\item Studied the ability of the LDMX detector to perform electron-nucleus cross-section
                %      \\ measurements useful to improving the modeling of neutrino-nucleus 
                %      interactions.
                \item Applied machine learning techniques to identify and veto 99.9999\% of rare 
                      photon-induced reactions expected to be the dominant background for LDMX.     
                      %(e.g. photo-nuclear) 
                %\item Use high performance computing techniques to produce large Monte Carlo data sets
                %      (~$\sim$10 billion per set) that were used to study the sensitivity of LDMX 
                %      to several physics scenarios.
                \item Played a leading role in the installation and commissioning of the upgraded Heavy Photon Search 
                      (HPS) silicon vertex tracker (SVT) including the integration of the rogue based data 
                      acquisition system and 
                      development of the monitoring GUI's using Display Manager.
                      %and the integration of its data acquisition
                      %system with Jefferson Lab's Hall-B central data acquisition system for the 2019 run.
                \item Conducted a resonance search for a new gauge boson (dark photon) in the mass range 
                      19 MeV/c$^2$ to 81 MeV/c$^2$ leading to the first physics publication by the
                      HPS experiment.
                %\item Mentored HPS graduate students and post-docs on machine learning, 
                %      analysis and data acquisition projects.
                %\item Mentored LDMX undergraduates, graduate students and post-docs on 
                %      machine learning, simulation and reconstruction projects.
            \end{itemize} 
                        } 
        \experienceentry{Santa Cruz Institute for Particle Physics}{Santa Cruz, CA}
                        {2011-2016}
                        {Graduate Student Researcher}
                        {   
            \begin{itemize}[label=\textcolor{indigodye}{$\circ$}, noitemsep, nolistsep, leftmargin=0.19\textwidth]
                \item Lead developer of a C++ analysis pipeline and statistical package
                      used to conduct a resonance search for a new fundamental particle.
                %$\item Utilized frequentist statistical analysis and maximum likelihood estimation to conduct a 
                %      resonance search for a new fundamental particle, heavy photon, thought to mediate dark matter 
                %      interactions.
                %\item Applied machine learning techniques  
                %      to reject trident and wide angle brem backgrounds.
                %using a Random Forest Algorithm in Scikit-Learn, boosting the 
                %signal/background fraction by 40\%.
                \item Co-creator of a Java data processing pipeline used to 
                      clean up, reconstruct, and visualize over 500 TB of noisy detector data. 
                %\item Developed Java front end used to load and retrieve greater than 100,000 
                 %      calibration constants from a MySQL database.
                \item Tested and commissioned the HPS SVT data acquisition system used for the 2015 and
                      2016 engineering runs. 
                      %used to continuously read out 23,004 channels at a rate of up to
                      %50 kHz.
                \item Qualified the performance of several components of the HPS
                      SVT including the front end readout boards and the silicon microstrip
                      modules at various stages of production.
                %\item Key member of team that assembled, installed and 
                %      commissioned the HPS SVT.
            \end{itemize} 
                      }
                      \newpage
        \experienceentry{Santa Cruz Institute for Particle Physics}{Santa Cruz, CA}
                        {2009-2011}
                        {Graduate Student Researcher}
                        {
            \begin{itemize}[label=\textcolor{indigodye}{$\circ$}, noitemsep, nolistsep, leftmargin=0.19\textwidth]
                \item Characterized the performance of the Long Shaping Time Front End readout chip at
                      various stages of development.
            \end{itemize}
                        } 
        \experienceentry{Department of Physics and Astronomy, California State University, Los Angeles}
                        {CA}
                        {2007-2009}
                        {Graduate Student Researcher}
                        {
            \begin{itemize}[label=\textcolor{indigodye}{$\circ$}, noitemsep, nolistsep, leftmargin=0.19\textwidth]
                %\item Developed C++ analysis used to optimize detector selection criteria using regression
                %      techniques and frequentist inference resulting in an improved understanding of the
                %      interaction of the proton in polyethylene.
                \item Used likelihood techniques to measure the form factor ratio, $G_{E_p}/G_{M_p}$, of
                      the proton.
            \end{itemize}
                        }  
        \experienceentry{Department of Physics and Astronomy, University of California, Irvine}
                        {CA}
                        {2005-2006}
                        {Undergraduate Researcher}
                        { 
            \begin{itemize}[label=\textcolor{indigodye}{$\circ$}, noitemsep, nolistsep, leftmargin=0.19\textwidth]
                \item Developed an analysis to measure the branching fraction for the extremely rare decay \\ 
                      $B\rightarrow e^+e^-$.
                \item Implemented and used a C++ based neural network to boost the identification of 
                      the particle decay $\Lambda \rightarrow p \pi^-$ by 10\%.
            \end{itemize}
                       } 
        %\experienceentry{2000-2001}
        %                {Mechanical Engineering Apprentice}
        %                {Nasa Dryden Flight Research Center}
        %                {Edwards, CA}
        %                {
        %                    \begin{itemize}[label=\textcolor{indigodye}{$\circ$}, noitemsep, nolistsep, leftmargin=0.19\textwidth]
        %                        \item Designed and constructed a device used to evaluate the skin-friction reduction 
        %                              of several Micro -Blowing Technique skins at supersonic speeds. 
        %                    \end{itemize}
        %                } 
    
    \cvsection{Skills}
        \skillsentry{Prog. Lang.}{C++, Python,  Java, MySQL. Familiar with JavaScript and Fortran.} 
        %\skillsentry{Tools}{%Geant4, 
        %                    git, \LaTeX, Linux, matplotlib, numpy, ROOT, RooFit, tensorflow, scipy, CMake}
        \skillsentry{Languages}{Fluent in English and Spanish}

    \cvsection{Appointments, Fellowships and Honors}
        \awardentry{2019}{Luis Alvarez Award for Best Experimental Research}{American Physical Society}
        \awardentry{2018}{Visiting Professor}{Università degli Studi di Sassari, Sassari, Italy}
        \awardentry{2012}{Margaret Burbidge Award for Best Experimental Research}{American Physical Society} 
        \awardentry{2011}{Regent's Fellowship}{University of California, Santa Cruz}
        \awardentry{2010}{GAANN Fellowship}{University of California, Santa Cruz}
        \awardentry{2009}{Special Recognition in Graduate Studies}
                         {California State University, Los Angeles}
        \awardentry{2009}{Margaziotis Award for Best Experimental Research}
                         {California State University, Los Angeles}
        \awardentry{2007-2009}{LSAMP Bridge to the Doctorate Fellowship}
                              {National Science Foundation} 
        \awardentry{2006}{California Alliance for Minority Participation Mentor of the Year}
                         {University of California, Irvine}
        \awardentry{2006}{Special Merit in Research}
                         {University of California, Irvine}
        \awardentry{2001-2002}{Chancellor's Leadership Scholar}
                         {University of California, Irvine}

    \cvsection{Leadership}
        \leadentry{2017-present}{Coordinator of the LDMX Software and Computing Working Group}
        \leadentry{2019-2021}{Member of the Heavy Photon Search Executive Committee}
        \leadentry{2016-2017}{Leader of the HPS Tracking Working Group}
        \leadentry{2015-2018}{Leader of the Resonance Search Working Group}

    %\cvsection{Teaching Experience}
    %    \teachingentry{2013-2015}{GRE Physics Bootcamp Instructor}
    %                  {Department of Physics, University of California, Santa Cruz}
    %                  {
    %                    \begin{itemize}[label=\textcolor{indigodye}{$\circ$}, noitemsep, nolistsep, leftmargin=0.19\textwidth]
    %                        \item Taught undergraduate level quantum mechanics.                                 
    %                      \end{itemize}
    %                  }

    %    \teachingentry{2009-2011}{Graduate Teaching Assistant}
    %                  {Department of Physics, University of California, Santa Cruz}
    %                  {
    %                    \begin{itemize}[label=\textcolor{indigodye}{$\circ$}, noitemsep, nolistsep, leftmargin=0.19\textwidth]
    %                        \item Physics 6A - Mechanics                                 
    %                        \item Physics 6B - Waves and Thermodynamics                  
    %                        \item Physics 6C - Electricity and Magnetism                
    %                      \end{itemize}
    %                  }

    %    \teachingentry{2007}{Graduate Teaching Assistant}
    %                  {Department of Physics and Astronomy, California State University, Los Angeles}
    %                  {  
    %                    \begin{itemize}[label=\textcolor{indigodye}{$\circ$}, noitemsep, nolistsep, leftmargin=0.19\textwidth]
    %                        \item Physics 211 - Classical Mechanics                      
    %                        \item Physics 213 - Electricity and Magnetism 
    %                    \end{itemize}
    %                  }
   
    \cvsection{Invited Talks}
        \vspace{-11pt}
        \begin{itemize}[label=\textcolor{indigodye}{$\circ$}, noitemsep, nolistsep, leftmargin=.03\textwidth]
            \item Moreno, O. \emph{Accelerator Based Searches for Dark Matter.}
                  Jefferson Lab Hall B Experimental Seminar. (2020). Newport News, VA.
            \item Moreno, O. \emph{Visible Dark Sector Probes.}
                  Light Dark Matter at Accelerators 2019. (2019). Venice, Italy. 
            \item Moreno, O. \emph{Accelerating Dark Matter}. 
                  Seminar given at the Dipartimento di Fisica, Università di Roma Tor Vergata.
                  (2018). Rome, Italy.
            \item Moreno, O. \emph{Accelerating Dark Matter}. 
                  Seminar given at the Dipartimento di Fisica, Università degli Studi di Torino.
                  (2018). Turin, Italy.  
            \item Moreno, O. \emph{Accelerating Dark Matter}. 
                  Seminar given at the Istituto Nazionale di Fisica Nucleare Genova.
                  (2018). Genoa, Italy. 
            \item Moreno, O. \emph{Probing the Dark World with Accelerators}. 
                  Seminar given at the Dipartamento di Chimica, Università degli Studi di Sassari.
                  (2018). Sassari, Italy. 
            \item Moreno, O. \emph{The Heavy Photon Search Experiment}. 
                  Fermi National Accelerator Laboratory LHC Physics Center Topic of the Week. (2018). Batavia, IL.
            \item Moreno, O. \emph{First Results from the Heavy Photon Search.}
                  Jefferson Lab Physics Seminar. (2017). Newport News, VA.
            \item Moreno, O. \emph{The Heavy Photon Search Experiment.}
                  U.S. Cosmic Visions: New Ideas in Dark Matter. (2017). College Park, MD.  
        \end{itemize}
        \vspace{11pt}

    \cvsection{Publications}
    \vspace{-22pt}

    \begin{thebibliography}{99}

      \bibitem{Eichlersmith:2022bit}
        T.~Eichlersmith{\it el al.}
        \emph{Simulation of dark bremsstrahlung in Geant4,}
        Comput. Phys. Commun. \textbf{287}, 108690 (2023)
        [arXiv:2211.03873 [hep-ph]].

      \bibitem{Bryngemark:2021pfy}
        L.~K.~Bryngemark{\it el al.}
        \emph{Building a Distributed Computing System for LDMX - Challenges of creating and operating a lightweight e-infrastructure for small-to-medium size accelerator experiments},
        EPJ Web Conf. \textbf{251}, 02038 (2021)
        [arXiv:2105.02977 [hep-ex]].

      \bibitem{Fadeyev:2020ydo}
        V.~Fadeyev {\it et al.} 
        \emph{Design and performance of silicon strip sensors with slim edges for HPS experiment},
        \\ Nucl. Instrum. Meth. A \textbf{969}, 163991 (2020)
    
      \bibitem{Ankowski:2019mfd}
        A.~M.~Ankowski {\it et al.}
        \emph{Lepton-Nucleus Cross Section Measurements for DUNE with the LDMX Detector},
        Phys. Rev. D \textbf{101}, no.5, 053004 (2020)  
        arXiv:1912.06140 [hep-ph].

      \bibitem{Akesson:2019iul} 
        T.~Åkesson {\it et al.} [LDMX Collaboration],
        \emph{A High Efficiency Photon Veto for the Light Dark Matter eXperiment},
        JHEP \textbf{04}, 003 (2020)
        arXiv:1912.05535 [hep-ex].

       \bibitem{Adrian:2018scb} 
         P.~H.~Adrian {\it et al.} [HPS Collaboration],
         \emph{Search for a dark photon in electroproduced $e^{+}e^{-}$ pairs with the Heavy Photon Search experiment at JLab},
         Phys.\ Rev.\ D {\bf 98}, no. 9, 091101 (2018)
         arXiv:1807.11530 [hep-ex].
        
       \bibitem{Akesson:2018vlm} 
         T.~Åkesson {\it et al.} [LDMX Collaboration],
         \emph{Light Dark Matter eXperiment (LDMX)},
            arXiv:1808.05219 [hep-ex].

       \bibitem{Puckett:2017flj} 
         A.~J.~R.~Puckett {\it et al.} [GEp-III Collaboration],
         \emph{Polarization Transfer Observables in Elastic Electron Proton Scattering at $Q^2 = $2.5, 5.2, 6.8, and 8.5 GeV$^2$},
         Phys.\ Rev.\ C {\bf 96}, no. 5, 055203 (2017)
         Erratum: [Phys.\ Rev.\ C {\bf 98}, no. 1, 019907 (2018)]
         arXiv:1707.08587 [nucl-ex].

       \bibitem{Puckett:2017egz} 
         A.~J.~R.~Puckett {\it et al.},
         \emph{Technical Supplement to "Polarization Transfer Observables in Elastic Electron-Proton Scattering at Q$^2$ = 2.5, 5.2, 6.8, and 8.5 GeV$^2$"},
         Nucl.\ Instrum.\ Meth.\ A {\bf 910}, 54 (2018)
         arXiv:1707.07750 [nucl-ex].

       \bibitem{Battaglieri:2014hga} 
         M.~Battaglieri {\it et al.},
         \emph{The Heavy Photon Search Test Detector},
         Nucl.\ Instrum.\ Meth.\ A {\bf 777}, 91 (2015)
         arXiv:1406.6115 [physics.ins-det].

       \bibitem{Luo:2011uy} 
         W.~Luo {\it et al.} [GEp-III and GEp2gamma Collaborations],
         \emph{Polarization components in $\pi^{0}$ photoproduction at photon energies up to 5.6 GeV},
         Phys.\ Rev.\ Lett.\  {\bf 108}, 222004 (2012)
         arXiv:1109.4650 [nucl-ex].

       \bibitem{Meziane:2010xc} 
         M.~Meziane {\it et al.} [GEp2gamma Collaboration],
         \emph{Search for effects beyond the Born approximation in polarization transfer observables in $\vec{e}p$ elastic scattering},
         Phys.\ Rev.\ Lett.\  {\bf 106}, 132501 (2011)
         arXiv:1012.0339 [nucl-ex].

      \bibitem{Puckett:2010ac} 
         A.~J.~R.~Puckett {\it et al.},
         \emph{Recoil Polarization Measurements of the Proton Electromagnetic Form Factor Ratio to Q$^2$ = 8.5 GeV$^2$},
         Phys.\ Rev.\ Lett.\  {\bf 104}, 242301 (2010)
         arXiv:1005.3419 [nucl-ex].

    \end{thebibliography}
    \nocite{*}

\end{document}
