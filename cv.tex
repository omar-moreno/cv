%
% cv.tex
% @author Omar Moreno
% @date March 01, 2016
%

\documentclass[11pt]{article}

%%%%%%%%%%%%%%%%
%   Packages   %
%%%%%%%%%%%%%%%%
\usepackage[usenames,dvipsnames]{xcolor}
\usepackage[top=1.0in, bottom=1.0in, left=0.5in, right=0.5in]{geometry}
\usepackage{enumitem}
\usepackage[none]{hyphenat}
\usepackage[hidelinks]{hyperref}
\usepackage{fontawesome}
\usepackage{changepage}

\pagenumbering{gobble}

% Use the font Computer Modern Sans Serif
\renewcommand*{\familydefault}{\sfdefault}

% Define Dim Gray color 
\definecolor{dimgray}{HTML}{696969}
\definecolor{indigodye}{HTML}{0D4F8B}

%
% resumesection : Command used to denote a new section of the CV
% 
% #1 Name of the section
%
\newcommand{\cvsection}[1] {
    \noindent
    \textcolor{indigodye}{\rule{.15\textwidth}{.1in} \hspace{0.01 \textwidth} \textbf{\Large{#1}}} \newline 
}

\newcommand{\educationentry}[6] { 
    \noindent
    \begin{minipage}[t]{0.15\textwidth} \begin{flushright} #1 \end{flushright} \end{minipage} \hspace{0.01\textwidth}
    \begin{minipage}[t]{0.84\textwidth} 
        \textbf{#2}, \emph{#3}, #4 \newline 
        #5 \newline
        Advisor: #6 \newline 
    \end{minipage}
}

\newcommand{\experienceentry}[5] { 
    \noindent
    \begin{minipage}[t]{0.15\textwidth} \begin{flushright} #1 \end{flushright} \end{minipage} \hspace{0.01\textwidth}
    \begin{minipage}[t]{0.84\textwidth} 
        \textbf{#2} \newline
        \emph{#3}, #4 
    \end{minipage} \\[0.01pt]
    #5 \vspace{11pt}
}

\newcommand{\skillsentry}[2] { 
    \noindent
    \begin{minipage}[t]{0.15\textwidth} \begin{flushright} #1 \end{flushright} \end{minipage} \hspace{0.01\textwidth}
    \begin{minipage}[t]{0.84\textwidth} #2 \end{minipage}
}

\newcommand{\awardentry}[3] { 
    \noindent
    \begin{minipage}[t]{0.15\textwidth} \begin{flushright} #1 \end{flushright} \end{minipage} \hspace{0.01\textwidth}
    \begin{minipage}[t]{0.84\textwidth} #2,  \emph{#3} 
    \end{minipage}
}

\newcommand{\teachingentry}[4] { 
    \noindent
    \begin{minipage}[t]{0.15\textwidth} \begin{flushright} #1 \end{flushright} \end{minipage} \hspace{0.01\textwidth}
    \begin{minipage}[t]{0.84\textwidth}
        \textbf{#2} \newline
        #3  
    \end{minipage} \\[0.01pt]
    #4 \vspace{11pt}
}


\renewcommand{\section}[2]{}

\begin{document}
    
    %%%%%%%%%%%%%%
    %   Header   %
    %%%%%%%%%%%%%%
    \noindent
    \begin{minipage}[c]{0.5\textwidth}
        \begin{flushleft}
            \Huge{Omar Moreno} \newline
            \Large{\textcolor{dimgray}{Curriculum Vitae}}
        \end{flushleft}
    \end{minipage}
    \begin{minipage}[c]{0.50\textwidth}
        \begin{flushright}
            \color{dimgray} \em
            Sunnyvale, CA           \\
            \faMobilePhone \hspace{1pt} +1 (562) 396-1622       \\
            \faEnvelope \hspace{1pt} \href{mailto:omoreno1@ucsc.edu}{omoreno1@ucsc.edu}          \\
            \faLinkedin \hspace{1pt} \href{https://www.linkedin.com/in/omarmoreno2}{omarmoreno2} \\
            \faGithub \hspace{1pt} \href{https://github.com/omar-moreno}{omar-moreno}            \\
        \end{flushright}
    \end{minipage}

    \cvsection{Education}
        \educationentry{(Expected) 2016}
                       {Ph.D. in Physics}
                       {University of California at Santa Cruz}
                       {Santa Cruz, CA}
                       {Dissertation: Resonance Search for a New Gauge Boson}
                       {Bruce Schumm}
        \educationentry{2007-2009}
                       {M.Sc. in Physics}
                       {California State University, Los Angeles}
                       {Los Angeles, CA}
                       {Thesis: Measurement of the Analyzing Power for the Reactions $p + CH_{2}
                       \rightarrow X$ at a \\ Proton Momentum of 2.2032 GeV/c}
                       {Konrad Aniol} 
        \educationentry{2001-2006}
                       {B.Sc. in Applied Physics}
                       {University of California at Irvine}
                       {Irvine, CA}
                       {Thesis: Search for the $B\rightarrow e^+e^-$ as a Hint to Supersymmetry}
                       {David Kirkby}

    \cvsection{Research Experience}
        \experienceentry{2011-present}
                        {Graduate Student Researcher, Heavy Photon Search Collaboration}
                        {Santa Cruz Institute for Particle Physics}
                        {Santa Cruz, CA}
                        {   
                            \begin{itemize}[label=\textcolor{indigodye}{$\circ$}, noitemsep, nolistsep, leftmargin=0.19\textwidth]
                              \item Utilized frequentist statistical analysis and maximum likelihood estimation to conduct a 
                                    resonance search for a new fundamental particle, heavy photon, thought to mediate dark matter 
                                    interactions.
                              \item Applied machine learning techniques to optimize the selection of signal like 
                                    events using a Random Forest Algorithm in Scikit-Learn, boosting the 
                                    signal/background fraction by 40\%.
                              \item Co-creator of both a C++ and Java analysis pipeline used to process, clean up,
                                    and visualize over 10 TB of noisy data (detector output).
                              \item Developed Java front end used to load and retrieve greater than 100,000 
                                    calibration constants from a MySQL database.
                              \item Tested and commissioned the Heavy Photon Search (HPS) Silicon Vertex Tracker (SVT) data
                                    acquisition system used to continuously read out 23004 channels at a rate of up to
                                    50 kHz.
                              \item Characterized the performance of several components of the HPS SVT
                                    including the APV25 readout chips and silicon microstrip modules at various stages
                                    of production.
                              \item Key member of team that assembled, installed and commissioned the HPS SVT.
                          \end{itemize} 
                      } 
        \experienceentry{2009-2011}
                        {Graduate Student Researcher, International Linear Collider}
                        {Santa Cruz Institute for Particle Physics}
                        {Santa Cruz, CA}
                        {
                            \begin{itemize}[label=\textcolor{indigodye}{$\circ$}, noitemsep, nolistsep, leftmargin=0.19\textwidth]
                                \item Characterized the performance of the Long Shaping Time Front End readout chip at
                                      various stages of development.
                                \item Mentored several undergraduate students on various projects.
                            \end{itemize}
                        } 
        \experienceentry{2007-2009}
                        {Graduate Student Researcher, GEP-III Collaboration}
                        {Department of Physics and Astronomy, California State University, Los Angeles}
                        {Los Angeles, CA}
                        {
                            \begin{itemize}[label=\textcolor{indigodye}{$\circ$}, noitemsep, nolistsep, leftmargin=0.19\textwidth]
                                \item Developed C++ analysis used to optimize detector selection criteria using regression
                                      techniques and frequentist inference resulting in an improved understanding of the
                                      interaction of the proton in polyethylene.
                                \item Performed statistical analysis to measure the form factor ratio, $G_{E_p}/G_{M_p}$, of
                                      the proton using blind analysis techniques.
                            \end{itemize}
                        }  
        \experienceentry{2005-2006}
                        {Undergraduate Researcher, BaBar Collaboration}
                        {Department of Physics and Astronomy, University of California, Irvine}
                        {Irvine, CA}
                        { 
                            \begin{itemize}[label=\textcolor{indigodye}{$\circ$}, noitemsep, nolistsep, leftmargin=0.19\textwidth]
                                \item Developed analysis to measure the branching fraction for the extremely rare decay 
                                      $B\rightarrow e^+e^-$ using blind analysis and regression techniques.
                                \item Used a neural network to boost the identification of the particle decay 
                                      $\Lambda \rightarrow p \pi^-$ by 10\%.
                            \end{itemize}
                       } 
        \experienceentry{2000-2001}
                        {Mechanical Engineering Apprentice}
                        {Nasa Dryden Flight Research Center}
                        {Edwards, CA}
                        {
                            \begin{itemize}[label=\textcolor{indigodye}{$\circ$}, noitemsep, nolistsep, leftmargin=0.19\textwidth]
                                \item Designed and constructed a device used to evaluate the skin-friction reduction 
                                      of several Micro -Blowing Technique skins at supersonic speeds. 
                            \end{itemize}
                        } 
    
    \cvsection{Skills}
        \skillsentry{Prog. Languages}{Java, C++, C, Python, MySQL, XML, Mathematica. Familiar with with HTML5 and Fortran}
        \skillsentry{Tools}{Linux, NumPy, Matplotlib, scikit-learn, scipy, git, SVN, CMake, \LaTeX, ROOT, RooFit}
        \skillsentry{Languages}{Fluent in English and Spanish}

    \cvsection{Fellowships and Honors}
        \awardentry{2012}{Margaret Burbidge Award for Best Experimental Research}{American Physical Society} 
        \awardentry{2011}{GAANN Fellowship}{University of California, Santa Cruz}
        \awardentry{2009}{Special Recognition in Graduate Studies}
                         {California State University, Los Angeles}
        \awardentry{2009}{Margaziotis Award for Best Experimental Research}
                         {California State University, Los Angeles}
        \awardentry{2007-2009}{LSAMP Bridge to the Doctorate Fellowship}
                              {National Science Foundation} 
        \awardentry{2006}{California Alliance for Minority Participation Mentor of the Year}
                         {University of California, Irvine}
        \awardentry{2006}{Special Merit in Research}
                         {University of California, Irvine}
        \awardentry{2001-2002}{Chancellor's Leadership Scholar}
                         {University of California, Irvine}

    \cvsection{Teaching Experience}
        \teachingentry{2013-2015}{GRE Physics Bootcamp Instructor}
                      {Department of Physics, University of California, Santa Cruz}
                      {
                        \begin{itemize}[label=\textcolor{indigodye}{$\circ$}, noitemsep, nolistsep, leftmargin=0.19\textwidth]
                            \item Taught undergraduate level quantum mechanics.                                 
                          \end{itemize}
                      }

        \teachingentry{2009-2011}{Graduate Teaching Assistant}
                      {Department of Physics, University of California, Santa Cruz}
                      {
                        \begin{itemize}[label=\textcolor{indigodye}{$\circ$}, noitemsep, nolistsep, leftmargin=0.19\textwidth]
                            \item Physics 6A - Mechanics                                 
                            \item Physics 6B - Waves and Thermodynamics                  
                            \item Physics 6C - Electricity and Magnetism                
                          \end{itemize}
                      }

        \teachingentry{2007}{Graduate Teaching Assistant}
                      {Department of Physics and Astronomy, California State University, Los Angeles}
                      {  
                        \begin{itemize}[label=\textcolor{indigodye}{$\circ$}, noitemsep, nolistsep, leftmargin=0.19\textwidth]
                            \item Physics 211 - Classical Mechanics                      
                            \item Physics 212 - Waves and Thermodynamics                 
                            \item Physics 213 - Electricity and Magnetism 
                        \end{itemize}
                      }

    \cvsection{Publications}
    \vspace{-22pt}

    \bibliography{publications}
    \bibliographystyle{unsrt}
    \nocite{*}

\end{document}
